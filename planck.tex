\documentclass{jsarticle}
\usepackage{amssymb,amsmath}
\usepackage{newtxtt}
\usepackage[utf8]{inputenc}
\newcommand{\pder}[2][]{\frac{\partial#1}{\partial#2}}
\newcommand{\dder}[2][]{\frac{\mathrm{d}#1}{\mathrm{d}#2}}
\newcommand{\half}{\frac{1}{2}}
\newcommand{\beq}{\begin{equation}}
\newcommand{\beql}[1]{\begin{equation}\label{#1}}
\newcommand{\eeq}{\end{equation}}
\newcommand{\eeqp}{\;\;\;.\end{equation}}
\newcommand{\eeqc}{\;\;\;,\end{equation}}
\date{\today}
\author{山田龍}
\title{プランク定数}
\begin{document}
\maketitle
\section{プランク定数}
光子のエネルギーと振動数を特徴づけるパラメーター
\beq
E = h \omega
\eeq

\section{ディラック定数}
\beq
h = 2 \pi \hbar
\eeq
$\hbar = 6.5 \times 10^{-16} eV \cdot s$
\section{歴史的経緯}
\subsection{黒体放射}
\subsection{光電効果および光量子仮説}
\subsection{ミリカンの実験}
\subsection{ダイオードで調べる}
\section{量子論において}
$h \sim 0$の極限で古典力学に一致する。
\subsection{シュレディンガー方程式からハミルトンヤコビ}
\end{document}

