\documentclass{jsarticle}
\usepackage{amssymb,amsmath}
\usepackage{newtxtt}
\usepackage[utf8]{inputenc}
\newcommand{\pder}[2][]{\frac{\partial#1}{\partial#2}}
\newcommand{\dder}[2][]{\frac{\mathrm{d}#1}{\mathrm{d}#2}}
\newcommand{\half}{\frac{1}{2}}
\newcommand{\beq}{\begin{equation}}
\newcommand{\beql}[1]{\begin{equation}\label{#1}}
\newcommand{\eeq}{\end{equation}}
\newcommand{\eeqp}{\;\;\;.\end{equation}}
\newcommand{\eeqc}{\;\;\;,\end{equation}}
\date{\today}
\author{山田龍}
\title{プランク定数}
\begin{document}
\maketitle
\section{プランク定数}
光子のエネルギーと振動数を特徴づけるパラメーター
\beq
E = h \omega
\eeq
$h = 4.1 \times 10^{-15} eV \cdot s = 6.6 \times 10^{-34} J \cdot s$
\section{ディラック定数}
換算プランク定数とも呼ばれる。角速度とエネルギーの関係を与える。
\beq
h = 2 \pi \hbar
\eeq
$\hbar = 6.5 \times 10^{-16} eV \cdot s = 1.0 \times 10^{-34} J \cdot s$
\section{歴史的経緯}
\subsection{黒体放射}
温度Tの平衡状態において、エネルギー量子$E =nh\nu$を仮定してEの期待値を計算する。
\begin{align}
    <E> &= \sum nh\nu exp(- nh\nu) / Z\\
    \sum exp(- nh\nu \beta) &=  \frac{e^{-h\nu \beta}}{e^{hv \beta} - 1}\\
    \sum nh\nu exp(- nh\nu\ beta) &= \dder[]{\beta} (\sum exp(-nh\nu \beta)) = \frac{h\nu e^{-h\nu \beta}}{(e^{hv \beta} - 1)^2}\\
    <E> &= \frac{h\nu}{e^{hv \beta} - 1}
\end{align}
黒体放射のデータからのフィッティングでプランク定数を知ることができる。
\subsection{光電効果および光量子仮説}
\subsection{ミリカンの実験}
ミリカンの実験では、光電効果のグラフの傾きからプランク定数を求める。
\beq
eV = h\mu - W
\eeq
からわかる。(ちなみに光電吸収の散乱断面積の大きさは$Z^5$)
\subsection{ダイオードで調べる}
発光ダイオードにおいては禁制帯域が波長によって変化する。
\beq 
h \nu = eV
\eeq
$\nu = c / \lambda$であるから、
\beq
h = eV \times \frac{\lambda}{c}
\eeq
ここで、複数の波長に対してLEDが発行し始める閾値を求めてやれば、一つの波長において計算した値より精度良く求められる。
\section{量子論において}
$h \sim 0$の極限で古典力学に一致する。
\subsection{シュレディンガー方程式からハミルトンヤコビ}
\end{document}

