\documentclass{jsarticle}
\usepackage{amssymb,amsmath}
\usepackage{newtxtt}
\usepackage[utf8]{inputenc}
\newcommand{\pder}[2][]{\frac{\partial#1}{\partial#2}}
\newcommand{\dder}[2][]{\frac{\mathrm{d}#1}{\mathrm{d}#2}}
\newcommand{\ppder}[2][]{\frac{\partial^2#1}{{\partial#2}^2}}
\newcommand{\half}{\frac{1}{2}}
\newcommand{\beq}{\begin{equation}}
\newcommand{\beql}[1]{\begin{equation}\label{#1}}
\newcommand{\eeq}{\end{equation}}
\newcommand{\eeqp}{\;\;\;.\end{equation}}
\newcommand{\eeqc}{\;\;\;,\end{equation}}
\date{\today}
\author{山田龍}
\title{アボガドロ定数}
\begin{document}
\maketitle
\subsection{定義}

物質量1molを構成する粒子
\beq
6.23 \times 10^{23}
\eeq
2019まで\\
0.012kgも炭素12に含まれる原子と等しい数の構成要素を含む径の物質量として定義されていた。\\
2019以降\\
6.022 14076に10の23乗をかけた要素粒子または要素粒子の集合体\\
キログラムに依存しなくなった
\section{測定}
\subsection{ブラウン運動}
\subsection{ファラデー定数と素電荷の比}
\subsection{プロトンの核磁気回転}
\subsection{X線回折と決勝の密度}
\subsection{X線と光干渉計を組み合わせた実験}
単結晶シリコンを使って格子定数を求める。$Si^{28}$の単結晶の単位格子には8つの原子が含まれる。格子の一辺の長さは格子定数と呼ばれ$a$で書かれる。
アボガドロ数は、
\beq
    N_A = \frac{8V}{a^3} \frac{M}{m}
\eeq
$M$はターゲットのモル質量。$m$はターゲットの質量。$M$はシリコンの同位体のモル質量を我々は知っているので、同位体の割合を推定すれば得られる。また、相対質量の計測精度は絶対質量の計測精度よりよいという利点もある。格子定数はX線干渉計を使って得る。
\section{プランク定数との関係}
\end{document}

